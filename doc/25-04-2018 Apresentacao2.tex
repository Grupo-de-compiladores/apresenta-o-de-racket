\documentclass{beamer}

\usetheme[
  outer/progressbar=frametitle
]{metropolis}

\usepackage[portuguese]{babel}
\usepackage[utf8]{inputenc}

\hypersetup{colorlinks=true,urlcolor=blue,linkcolor=blue,citecolor=blue}

\title[Compiladores 2018.1]{2$^{a.}$ Apresentação do compilador Dante}
\author[Abrev.]{Autores: João Pedro Abreu, Luis Freitas e Raphael Leardini} 

\institute[UFF]{Universidade Federal Fluminense}

\date{Data: 25/04/2018} 

\begin{document}

% ---

\begin{frame}[plain]

\titlepage

\end{frame}

% ---

\begin{frame}{PEG}
Ao detectar uma regra PEG, nós identificamos as partes importantes:
\begin{itemize}
	\item (Ex: (define-peg virg (and spaces "," spaces)))
	\item (Ex: (define-peg atribuicao (and (name t1 variable) spaces (or ":=" "=") spaces (name t2 (or boolExp aritExp string))) (atrib t2 t1))) 
\end{itemize}
e criamos uma árvore sintática do programa inteiro.
\end{frame}

% ---

\begin{frame}{Transformação}
Ao contrário dos outros grupos, fizemos duas transformações de árvores sintáticas.

A primeira foi de símbolos para uma línguagem abstrata que não era o BPLC.

(Ex: "4+4" transforma em (soma 4 4))

A segunda foi dessa línguagem abstrata para o BPLC.

(Ex: (soma 4 4) se transforma em (add 4 4))

A pergunta então se torna para que fazer isso?

\end{frame}

% ---
\begin{frame}{Transformação}
Motivos:
\begin{itemize}

   \item Mais fácil entendimento do código para os membros do grupo.
   \item Potencial de portabilidade desse mesmo compilador para outra linguagem.
   
\end{itemize}
Levando em conta um único contra, que é o tempo, que acabou não sendo problema grande nessa parte do projeto.
\end{frame}

%---
\begin{frame}{SMC}
/Coisas sobre SMC/



\end{frame}

%---
\begin{frame}{Dificuldades}
Com relação a PEG a maior dificuldade encontrada foi definir todas as possibilidades de uma sequência (Ex: "2+2" ou 
"2 +2" ou "2+ 2" ou "2 + 2")

Para a transformação a maior dificuldade foi a atribuição. Devido o tipo ID que a função assign do BPLC realiza, o responsável por essa transformação teve problemas.

\end{frame}

\end{document}
