\documentclass{beamer}

\usetheme[
  outer/progressbar=frametitle
]{metropolis}

\usepackage[portuguese]{babel}
\usepackage[utf8]{inputenc}

\hypersetup{colorlinks=true,urlcolor=blue,linkcolor=blue,citecolor=blue}

\title[Compiladores 2018.1]{1$^{a.}$ Apresentação do compilador Dante P2}
\author[Abrev.]{Autores: João Pedro Abreu, Luis Freitas e Raphael Leardini} 

\institute[UFF]{Universidade Federal Fluminense}

\date{Data: 16/05/2018} 

\begin{document}

% ---

\begin{frame}[plain]

\titlepage

\end{frame}

% ---

\begin{frame}{Parser}
\begin{verbatim}

(define-peg statement (or declaracao comando))


(define-peg bloco (and wordSeparator (name t1 declaracao) wordSeparator (name t2 comando)) (blk t1 t2))


(define-peg loop (and
"while" spaces (name condicao boolExp) wordSeparator "do" spaces "{" wordSeparator (name corpo (or bloco comando)) wordSeparator "}") (whileDo 
condicao corpo))

\end{verbatim}

\end{frame}

\begin{frame}{Bloco}

Quando o SMC encontra um bloco, ele salva o ambiente atual na pilha de valores, destroi o bloco e coloca no final do comando do bloco 'blk, para saber quando deve sair do bloco, ou seja, restaurar o contexto anteriormente salvo.

(blk a b) $\longrightarrow$ (a b 'blk)
\end{frame}


\begin{frame}{Ambiente}

O ambiente é um hash cujas chaves são strings e os valores são (U Number Boolean Loc), onde Loc é uma estrutura
que contém apenas um número como membro, para diferenciar variaveis de constantes.

\end{frame}


\end{document}
