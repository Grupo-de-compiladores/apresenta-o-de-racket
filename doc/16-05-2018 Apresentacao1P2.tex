\documentclass{beamer}

\usetheme[
  outer/progressbar=frametitle
]{metropolis}

\usepackage[portuguese]{babel}
\usepackage[utf8]{inputenc}

\hypersetup{colorlinks=true,urlcolor=blue,linkcolor=blue,citecolor=blue}

\title[Compiladores 2018.1]{1$^{a.}$ Apresentação do compilador Dante P2}
\author[Abrev.]{Autores: João Pedro Abreu, Luis Freitas e Raphael Leardini} 

\institute[UFF]{Universidade Federal Fluminense}

\date{Data: 16/05/2018} 

\begin{document}

% ---

\begin{frame}[plain]

\titlepage

\end{frame}

% ---

\begin{frame}{Parser}
\begin{verbatim}

(define-peg statement (or declaracao comando))


(define-peg bloco (and wordSeparator (name t1 declaracao) wordSeparator (name t2 comando)) (blk t1 t2))


(define-peg loop (and
"while" spaces (name condicao boolExp) wordSeparator "do" spaces "{" wordSeparator (name corpo (or bloco comando)) wordSeparator "}") (whileDo 
condicao corpo))

\end{verbatim}

\end{frame}

\begin{frame}{SMC}
O (S,M,C) foi extendido para (E,S,M,C,L) sendo o E o ambiente e o L a lista de localizações. Esta lista é utilizada, quando do retorno do bloco, para limpar a memoria das
variaveis criadas internamente. Atualmente a única parte não-funcional(entenda funcional como o paradigma) do codigo é a execução do print e do exit, mas isso poderia ser
resolvido extendendo para um (O,E,S,M,C,L) onde o O é uma lista de efeitos colaterais, como descrito na especificação(não realizado ainda).
\end{frame}

\begin{frame}{SMC - Problemas}
Assim como na primeira parte, não houveram percalços significativos na execução desta. O smc basicamente se tornou uma tradução assistida da especificação, sem necessidade
de entender o que se esta fazendo(nós entendemos).
\end{frame}

\begin{frame}{SMC}
Toda operação que "altera" o ambiente e a memoria foi encapsulada no modulo "ambiente.rkt",fazendo com que o smcEval seja um codigo de reescrita quase puro e simples.
Exatamente por isso podemos extender essa implementação para a utilização de uma função de avaliação, a qual aparece na especificação como "val" e igualmente serve
para devolver o valor ou o tipo, fazendo com que, sem uma única linha adicionada, o smcEval possa fazer type-checking. O único porém seria a avaliação de loops, o qual
exigiria algum contexto ou alguma tatica ainda não plenamente compreendida.
\end{frame}

\begin{frame}{SMC}
Toda operação que "altera" o ambiente e a memoria foi encapsulada no modulo "ambiente.rkt",fazendo com que o smcEval seja um codigo de reescrita quase puro e simples.
Exatamente por isso podemos extender essa implementação para a utilização de uma função de avaliação, a qual aparece na especificação como "val" e igualmente serve
para devolver o valor ou o tipo, fazendo com que, sem uma única linha adicionada, o smcEval possa fazer type-checking. O único porém seria a avaliação de loops, o qual
exigiria algum contexto ou alguma tatica ainda não plenamente compreendida.
\end{frame}

\begin{frame}{Bloco}

Quando o SMC encontra um bloco, ele salva o ambiente atual e a atual lista de localizações na pilha de valores, destroi o bloco e coloca no final do comando do bloco
'blk, para saber quando deve sair do bloco, ou seja, restaurar o contexto anteriormente salvo e limpar a memoria. Cada operação de criação de variaveis dentro de um
bloco adiciona a localização desta variavel na lista de localizações. (i.e cada ref faz um cons, cada 'blk faz um free)
\end{frame}


\begin{frame}{Ambiente}

O ambiente é um hash cujas chaves são strings(identificadores) e os valores são (U Number Boolean Loc), onde Loc é uma estrutura que contém apenas um número como membro,
para diferenciar variaveis de constantes.

\end{frame}


\end{document}
